\documentclass{article}
\usepackage{natbib}
\usepackage{hyperref}

%==============================================================================

\title{Example Usage of \texttt{apalike-ejor} \textsc{Bib\TeX} Style}
\author{Adam Rumpf\footnote{\url{https://github.com/adam-rumpf}}}
\date{June 20, 2021}

%==============================================================================

\begin{document}

\maketitle

\section{Introduction}

This example document demonstrates the usage of the \texttt{apalike-ejor} style, which was developed for formatting bibliographies in accordance with the \textit{European Journal of Operational Research} \href{https://www.elsevier.com/journals/european-journal-of-operational-research/0377-2217/guide-for-authors}{style guide} (as it appeared on June 18, 2021).

The \textsc{Bib\TeX} style file \texttt{apalike-ejor.bst} is a fork of \texttt{apalike.bst}, originally by Oren Patashnik, and mostly follows the standard APA style. Among other edits hyperlinked URL and DOI fields have been added, which requires the usage of the \texttt{hyperref} package.

In order to use \texttt{apalike-ejor}, make sure that the \texttt{apalike.ejor.bst} file is located somewhere where your compiler can find it, then compile your document with \textsc{Bib\TeX} a couple of times followed by \LaTeX\ a couple of times. At minimum the following should be included in your main \texttt{.tex} file:

\begin{verbatim}
\documentclass{article}
\usepackage{natbib} % required for inline citations
\usepackage{hyperref} % required if utilizing url or doi fields

\begin{document}
%%%
%%% include the main body of the document here
%%%

\bibliographystyle{apalike-ejor}
\bibliography{example-references} % replace with your .bib file

%%%
%%% include any text following the bibliography here
%%%
\end{document}
\end{verbatim}

\section{Datasets}

The \textit{EJOR} style guide requests that dataset references include ``[dataset]'' at the beginning of the reference (this is only for the manuscript, and does not appear in the published article). Since datasets are usually handled as \texttt{@misc} entries, a new \texttt{dataset} field is recognized by \texttt{apalike-ejor}.

Within a \texttt{.bib} file, including any string besides ``0'' or ``false'' (case-insensitive) in a \texttt{@misc} entry's \texttt{dataset} field results in its reference beginning with the word ``[dataset]''. Including ``0'' or ``false'', or excluding the \texttt{dataset} field, results in a normal \texttt{@misc} citation.

\section{Examples}

\begin{itemize}
	\item Article: \cite{kinney2005}
	\item Book: \cite{ahuja1993}
	\item Conference Proceedings: \cite{cavdaroglu2010}
	\item Dataset: \cite{kaul2021}
	\item Web: \cite{rumpf2019}
	\item Thesis: \cite{schmocker2006}
\end{itemize}

\section*{Acknowledgements}

Thank you to Oren Patashnik for writing the original \texttt{apalike} style file on which this was based. This modification of \texttt{apalike}, and the other files in this package, were largely based on the \texttt{zootaxa-bst} package by Gustavo A.\ Ballen, and seeing the \texttt{zootaxa-bst} repository on \href{https://github.com/gaballench/zootaxa-bst}{GitHub} is what originally gave me the motivation to make this slapdash and kludgy personal tool available to the public. Thank you also to the \href{https://tex.stackexchange.com/}{\LaTeX\ Stack Exchange} community, whose guidance proved invaluable in figuring out the basics of the \texttt{.bst} format, as well as to the \href{https://www.ctan.org/}{CTAN} community for making so much \TeX\ material openly available. Finally thank you to my graduate advisor Hemanshu Kaul, without whom I would never have been in a position to submit a paper to \textit{EJOR} in the first place.

\bibliographystyle{apalike-ejor}
\bibliography{example-references}

\end{document}
