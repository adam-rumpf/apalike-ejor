\documentclass{article}
\usepackage{natbib}
\usepackage{hyperref}

%==============================================================================

\title{Example Usage of \texttt{apalike-ejor} \textsc{Bib\TeX} Style}
\author{Adam Rumpf\footnote{\url{https://github.com/adam-rumpf}}}
\date{June 21, 2021}

%==============================================================================

\begin{document}

\maketitle

\section*{Introduction}

This example document demonstrates the usage of the \href{https://www.ctan.org/pkg/apalike-ejor}{\texttt{apalike-ejor}} style, which was developed for formatting bibliographies in accordance with the \textit{European Journal of Operational Research} \href{https://www.elsevier.com/journals/european-journal-of-operational-research/0377-2217/guide-for-authors}{style guide} (as it appeared in June 2021). The \textsc{Bib\TeX} style file \texttt{apalike-ejor.bst} is a fork of \href{https://www.bibtex.com/s/bibliography-style-base-apalike/}{\texttt{apalike.bst}}, originally by Oren Patashnik, and mostly follows standard \href{https://apastyle.apa.org/style-grammar-guidelines/references/examples}{APA style}.

In order to use \texttt{apalike-ejor}, make sure that the \texttt{apalike.ejor.bst} file is located somewhere where your compiler can find it, then compile your document with \textsc{Bib\TeX} a couple of times followed by \LaTeX\ a couple of times. At minimum the following should be included in your main \texttt{.tex} file:

\begin{verbatim}
\documentclass{article}
\usepackage{natbib} % required for inline citations
\usepackage{hyperref} % required if utilizing url or doi fields

\begin{document}
%%%
%%% include the main body of the document here
%%%

\bibliographystyle{apalike-ejor}
\bibliography{example-references} % replace with your .bib file

%%%
%%% include any text following the bibliography here
%%%
\end{document}
\end{verbatim}

\section*{URL and DOI Fields}

Support for \texttt{url} and \texttt{doi} fields has been added for all reference types. Both produce a hyperlinked URL within the reference, which requires the use of the \href{https://ctan.org/pkg/hyperref}{\texttt{hyperref}} package. For \texttt{doi} entries only the DOI name, itself, should be included, which will automatically be appended to a \url{https://doi.org/} link. For \texttt{url} entries the entire URL should be included. If both the \texttt{doi} and \texttt{url} fields are filled, only the \texttt{doi} link is created.

\section*{Datasets}

The \textit{EJOR} style guide requests that dataset references include ``[dataset]'' at the beginning of the reference (this is only for the manuscript, and does not appear in the published article). Since datasets are usually handled as \texttt{@misc} entries, a new \texttt{dataset} field is recognized by \texttt{apalike-ejor}.

Within a \texttt{.bib} file, including any string besides ``0'' or ``false'' (case insensitive) in a \texttt{@misc} entry's \texttt{dataset} field results in its reference beginning with the word ``[dataset]''. Including ``0'' or ``false'', or excluding the \texttt{dataset} field, results in a normal \texttt{@misc} citation. The \texttt{kaul2021} and \texttt{rumpf2019} references below provide an example of one dataset and one non-dataset \texttt{@misc} entry.

\section*{Example References}

The following entries are included in this repository's example reference file \texttt{example-references.bib}:

\begin{verbatim}
@book{ahuja1993,
address = {Englewood Cliffs, NJ},
author = {Ahuja, R K and Magnanti, T L and Orlin, J B},
chapter = {16},
edition = {1st},
publisher = {Prentice Hall},
title = {{Network Flows: Theory, Algorithms, and Applications}},
year = {1993}
}
@article{kinney2005,
author = {Kinney, R and Crucitti, P and Albert, R and Latora, V},
doi = {10.1140/epjb/e2005-00237-9},
journal = {European Physical Journal B},
pages = {101--106},
title = {{Modeling cascading failures in the North American power
          grid}},
volume = {46},
year = {2005}
}
\end{verbatim}
\newpage
\begin{verbatim}
@inproceedings{cavdaroglu2010,
author = {Cavdaroglu, B and Nurre, S G and Mitchell, J E and
          Sharkey, T C and Wallace, W A},
booktitle = {Vulnerability, Uncertainty, and Risk: Analysis,
             Modeling, and Management},
editor = {Ayyub, B M},
pages = {171--179},
publisher = {American Society of Civil Engineers},
title = {{Decomposition Methods for Restoring Infrastructure
          Systems}},
url = {https://ascelibrary.org/doi/10.1061/41170(400)21},
year = {2010}
}
@misc{kaul2021,
author = {Kaul, Hemanshu and Rumpf, Adam},
dataset = {1},
doi = {10.17632/ptzc7jxhmn.1},
howpublished = {Mendeley Data, V1},
title = {{A linear input dependence model for interdependent
          networks}},
url = {https://data.mendeley.com/datasets/ptzc7jxhmn/1},
year = {2021}
}
@misc{rumpf2019,
author = {Rumpf, Adam},
note = {Accessed May 8, 2020},
title = {{MCNFLI Computational Trials}},
url = {https://github.com/adam-rumpf/mcnfli-trials},
year = {2019}
}
@phdthesis{schmocker2006,
address = {London, England},
author = {Schm\"ocker, J-D},
school = {Imperial College London},
title = {{Dynamic Capacity Constrained Traffic Assignment}},
year = {2006}
}
\end{verbatim}

\noindent Their inline references are typeset as follows:\\

Article: \cite{kinney2005}

Book: \cite{ahuja1993}

Conference Proceedings: \cite{cavdaroglu2010}

Dataset: \cite{kaul2021}

Web: \cite{rumpf2019}

Thesis: \cite{schmocker2006}

Applying the \texttt{apalike-ejor} style results in the following References section:

\bibliographystyle{apalike-ejor}
\bibliography{example-references}

\section*{Acknowledgements}

Thank you to Oren Patashnik for writing the original \texttt{apalike} style file on which this was based. This modification of \texttt{apalike}, and the other files in this package, were largely based on the \href{https://www.ctan.org/pkg/zootaxa-bst}{\texttt{zootaxa-bst}} package by Gustavo A.\ Ballen, and seeing the \texttt{zootaxa-bst} repository on \href{https://github.com/gaballench/zootaxa-bst}{GitHub} is what originally gave me the motivation to make this slapdash and kludgy personal tool available to the public. Thank you also to the \href{https://tex.stackexchange.com/}{\LaTeX\ Stack Exchange} community, whose guidance proved invaluable in figuring out the basics of the \texttt{.bst} format, as well as to the \href{https://www.ctan.org/}{CTAN} community for making so much \TeX\ material openly available. Finally thank you to my graduate advisor Hemanshu Kaul, without whom I would never have been in a position to submit a paper to \textit{EJOR} in the first place.

\end{document}
